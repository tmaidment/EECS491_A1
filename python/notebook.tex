
% Default to the notebook output style

    


% Inherit from the specified cell style.




    
\documentclass[11pt]{article}

    
    
    \usepackage[T1]{fontenc}
    % Nicer default font (+ math font) than Computer Modern for most use cases
    \usepackage{mathpazo}

    % Basic figure setup, for now with no caption control since it's done
    % automatically by Pandoc (which extracts ![](path) syntax from Markdown).
    \usepackage{graphicx}
    % We will generate all images so they have a width \maxwidth. This means
    % that they will get their normal width if they fit onto the page, but
    % are scaled down if they would overflow the margins.
    \makeatletter
    \def\maxwidth{\ifdim\Gin@nat@width>\linewidth\linewidth
    \else\Gin@nat@width\fi}
    \makeatother
    \let\Oldincludegraphics\includegraphics
    % Set max figure width to be 80% of text width, for now hardcoded.
    \renewcommand{\includegraphics}[1]{\Oldincludegraphics[width=.8\maxwidth]{#1}}
    % Ensure that by default, figures have no caption (until we provide a
    % proper Figure object with a Caption API and a way to capture that
    % in the conversion process - todo).
    \usepackage{caption}
    \DeclareCaptionLabelFormat{nolabel}{}
    \captionsetup{labelformat=nolabel}

    \usepackage{adjustbox} % Used to constrain images to a maximum size 
    \usepackage{xcolor} % Allow colors to be defined
    \usepackage{enumerate} % Needed for markdown enumerations to work
    \usepackage{geometry} % Used to adjust the document margins
    \usepackage{amsmath} % Equations
    \usepackage{amssymb} % Equations
    \usepackage{textcomp} % defines textquotesingle
    % Hack from http://tex.stackexchange.com/a/47451/13684:
    \AtBeginDocument{%
        \def\PYZsq{\textquotesingle}% Upright quotes in Pygmentized code
    }
    \usepackage{upquote} % Upright quotes for verbatim code
    \usepackage{eurosym} % defines \euro
    \usepackage[mathletters]{ucs} % Extended unicode (utf-8) support
    \usepackage[utf8x]{inputenc} % Allow utf-8 characters in the tex document
    \usepackage{fancyvrb} % verbatim replacement that allows latex
    \usepackage{grffile} % extends the file name processing of package graphics 
                         % to support a larger range 
    % The hyperref package gives us a pdf with properly built
    % internal navigation ('pdf bookmarks' for the table of contents,
    % internal cross-reference links, web links for URLs, etc.)
    \usepackage{hyperref}
    \usepackage{longtable} % longtable support required by pandoc >1.10
    \usepackage{booktabs}  % table support for pandoc > 1.12.2
    \usepackage[inline]{enumitem} % IRkernel/repr support (it uses the enumerate* environment)
    \usepackage[normalem]{ulem} % ulem is needed to support strikethroughs (\sout)
                                % normalem makes italics be italics, not underlines
    

    
    
    % Colors for the hyperref package
    \definecolor{urlcolor}{rgb}{0,.145,.698}
    \definecolor{linkcolor}{rgb}{.71,0.21,0.01}
    \definecolor{citecolor}{rgb}{.12,.54,.11}

    % ANSI colors
    \definecolor{ansi-black}{HTML}{3E424D}
    \definecolor{ansi-black-intense}{HTML}{282C36}
    \definecolor{ansi-red}{HTML}{E75C58}
    \definecolor{ansi-red-intense}{HTML}{B22B31}
    \definecolor{ansi-green}{HTML}{00A250}
    \definecolor{ansi-green-intense}{HTML}{007427}
    \definecolor{ansi-yellow}{HTML}{DDB62B}
    \definecolor{ansi-yellow-intense}{HTML}{B27D12}
    \definecolor{ansi-blue}{HTML}{208FFB}
    \definecolor{ansi-blue-intense}{HTML}{0065CA}
    \definecolor{ansi-magenta}{HTML}{D160C4}
    \definecolor{ansi-magenta-intense}{HTML}{A03196}
    \definecolor{ansi-cyan}{HTML}{60C6C8}
    \definecolor{ansi-cyan-intense}{HTML}{258F8F}
    \definecolor{ansi-white}{HTML}{C5C1B4}
    \definecolor{ansi-white-intense}{HTML}{A1A6B2}

    % commands and environments needed by pandoc snippets
    % extracted from the output of `pandoc -s`
    \providecommand{\tightlist}{%
      \setlength{\itemsep}{0pt}\setlength{\parskip}{0pt}}
    \DefineVerbatimEnvironment{Highlighting}{Verbatim}{commandchars=\\\{\}}
    % Add ',fontsize=\small' for more characters per line
    \newenvironment{Shaded}{}{}
    \newcommand{\KeywordTok}[1]{\textcolor[rgb]{0.00,0.44,0.13}{\textbf{{#1}}}}
    \newcommand{\DataTypeTok}[1]{\textcolor[rgb]{0.56,0.13,0.00}{{#1}}}
    \newcommand{\DecValTok}[1]{\textcolor[rgb]{0.25,0.63,0.44}{{#1}}}
    \newcommand{\BaseNTok}[1]{\textcolor[rgb]{0.25,0.63,0.44}{{#1}}}
    \newcommand{\FloatTok}[1]{\textcolor[rgb]{0.25,0.63,0.44}{{#1}}}
    \newcommand{\CharTok}[1]{\textcolor[rgb]{0.25,0.44,0.63}{{#1}}}
    \newcommand{\StringTok}[1]{\textcolor[rgb]{0.25,0.44,0.63}{{#1}}}
    \newcommand{\CommentTok}[1]{\textcolor[rgb]{0.38,0.63,0.69}{\textit{{#1}}}}
    \newcommand{\OtherTok}[1]{\textcolor[rgb]{0.00,0.44,0.13}{{#1}}}
    \newcommand{\AlertTok}[1]{\textcolor[rgb]{1.00,0.00,0.00}{\textbf{{#1}}}}
    \newcommand{\FunctionTok}[1]{\textcolor[rgb]{0.02,0.16,0.49}{{#1}}}
    \newcommand{\RegionMarkerTok}[1]{{#1}}
    \newcommand{\ErrorTok}[1]{\textcolor[rgb]{1.00,0.00,0.00}{\textbf{{#1}}}}
    \newcommand{\NormalTok}[1]{{#1}}
    
    % Additional commands for more recent versions of Pandoc
    \newcommand{\ConstantTok}[1]{\textcolor[rgb]{0.53,0.00,0.00}{{#1}}}
    \newcommand{\SpecialCharTok}[1]{\textcolor[rgb]{0.25,0.44,0.63}{{#1}}}
    \newcommand{\VerbatimStringTok}[1]{\textcolor[rgb]{0.25,0.44,0.63}{{#1}}}
    \newcommand{\SpecialStringTok}[1]{\textcolor[rgb]{0.73,0.40,0.53}{{#1}}}
    \newcommand{\ImportTok}[1]{{#1}}
    \newcommand{\DocumentationTok}[1]{\textcolor[rgb]{0.73,0.13,0.13}{\textit{{#1}}}}
    \newcommand{\AnnotationTok}[1]{\textcolor[rgb]{0.38,0.63,0.69}{\textbf{\textit{{#1}}}}}
    \newcommand{\CommentVarTok}[1]{\textcolor[rgb]{0.38,0.63,0.69}{\textbf{\textit{{#1}}}}}
    \newcommand{\VariableTok}[1]{\textcolor[rgb]{0.10,0.09,0.49}{{#1}}}
    \newcommand{\ControlFlowTok}[1]{\textcolor[rgb]{0.00,0.44,0.13}{\textbf{{#1}}}}
    \newcommand{\OperatorTok}[1]{\textcolor[rgb]{0.40,0.40,0.40}{{#1}}}
    \newcommand{\BuiltInTok}[1]{{#1}}
    \newcommand{\ExtensionTok}[1]{{#1}}
    \newcommand{\PreprocessorTok}[1]{\textcolor[rgb]{0.74,0.48,0.00}{{#1}}}
    \newcommand{\AttributeTok}[1]{\textcolor[rgb]{0.49,0.56,0.16}{{#1}}}
    \newcommand{\InformationTok}[1]{\textcolor[rgb]{0.38,0.63,0.69}{\textbf{\textit{{#1}}}}}
    \newcommand{\WarningTok}[1]{\textcolor[rgb]{0.38,0.63,0.69}{\textbf{\textit{{#1}}}}}
    
    
    % Define a nice break command that doesn't care if a line doesn't already
    % exist.
    \def\br{\hspace*{\fill} \\* }
    % Math Jax compatability definitions
    \def\gt{>}
    \def\lt{<}
    % Document parameters
    \title{Assignment 1 - tdm47}
    
    
    

    % Pygments definitions
    
\makeatletter
\def\PY@reset{\let\PY@it=\relax \let\PY@bf=\relax%
    \let\PY@ul=\relax \let\PY@tc=\relax%
    \let\PY@bc=\relax \let\PY@ff=\relax}
\def\PY@tok#1{\csname PY@tok@#1\endcsname}
\def\PY@toks#1+{\ifx\relax#1\empty\else%
    \PY@tok{#1}\expandafter\PY@toks\fi}
\def\PY@do#1{\PY@bc{\PY@tc{\PY@ul{%
    \PY@it{\PY@bf{\PY@ff{#1}}}}}}}
\def\PY#1#2{\PY@reset\PY@toks#1+\relax+\PY@do{#2}}

\expandafter\def\csname PY@tok@w\endcsname{\def\PY@tc##1{\textcolor[rgb]{0.73,0.73,0.73}{##1}}}
\expandafter\def\csname PY@tok@c\endcsname{\let\PY@it=\textit\def\PY@tc##1{\textcolor[rgb]{0.25,0.50,0.50}{##1}}}
\expandafter\def\csname PY@tok@cp\endcsname{\def\PY@tc##1{\textcolor[rgb]{0.74,0.48,0.00}{##1}}}
\expandafter\def\csname PY@tok@k\endcsname{\let\PY@bf=\textbf\def\PY@tc##1{\textcolor[rgb]{0.00,0.50,0.00}{##1}}}
\expandafter\def\csname PY@tok@kp\endcsname{\def\PY@tc##1{\textcolor[rgb]{0.00,0.50,0.00}{##1}}}
\expandafter\def\csname PY@tok@kt\endcsname{\def\PY@tc##1{\textcolor[rgb]{0.69,0.00,0.25}{##1}}}
\expandafter\def\csname PY@tok@o\endcsname{\def\PY@tc##1{\textcolor[rgb]{0.40,0.40,0.40}{##1}}}
\expandafter\def\csname PY@tok@ow\endcsname{\let\PY@bf=\textbf\def\PY@tc##1{\textcolor[rgb]{0.67,0.13,1.00}{##1}}}
\expandafter\def\csname PY@tok@nb\endcsname{\def\PY@tc##1{\textcolor[rgb]{0.00,0.50,0.00}{##1}}}
\expandafter\def\csname PY@tok@nf\endcsname{\def\PY@tc##1{\textcolor[rgb]{0.00,0.00,1.00}{##1}}}
\expandafter\def\csname PY@tok@nc\endcsname{\let\PY@bf=\textbf\def\PY@tc##1{\textcolor[rgb]{0.00,0.00,1.00}{##1}}}
\expandafter\def\csname PY@tok@nn\endcsname{\let\PY@bf=\textbf\def\PY@tc##1{\textcolor[rgb]{0.00,0.00,1.00}{##1}}}
\expandafter\def\csname PY@tok@ne\endcsname{\let\PY@bf=\textbf\def\PY@tc##1{\textcolor[rgb]{0.82,0.25,0.23}{##1}}}
\expandafter\def\csname PY@tok@nv\endcsname{\def\PY@tc##1{\textcolor[rgb]{0.10,0.09,0.49}{##1}}}
\expandafter\def\csname PY@tok@no\endcsname{\def\PY@tc##1{\textcolor[rgb]{0.53,0.00,0.00}{##1}}}
\expandafter\def\csname PY@tok@nl\endcsname{\def\PY@tc##1{\textcolor[rgb]{0.63,0.63,0.00}{##1}}}
\expandafter\def\csname PY@tok@ni\endcsname{\let\PY@bf=\textbf\def\PY@tc##1{\textcolor[rgb]{0.60,0.60,0.60}{##1}}}
\expandafter\def\csname PY@tok@na\endcsname{\def\PY@tc##1{\textcolor[rgb]{0.49,0.56,0.16}{##1}}}
\expandafter\def\csname PY@tok@nt\endcsname{\let\PY@bf=\textbf\def\PY@tc##1{\textcolor[rgb]{0.00,0.50,0.00}{##1}}}
\expandafter\def\csname PY@tok@nd\endcsname{\def\PY@tc##1{\textcolor[rgb]{0.67,0.13,1.00}{##1}}}
\expandafter\def\csname PY@tok@s\endcsname{\def\PY@tc##1{\textcolor[rgb]{0.73,0.13,0.13}{##1}}}
\expandafter\def\csname PY@tok@sd\endcsname{\let\PY@it=\textit\def\PY@tc##1{\textcolor[rgb]{0.73,0.13,0.13}{##1}}}
\expandafter\def\csname PY@tok@si\endcsname{\let\PY@bf=\textbf\def\PY@tc##1{\textcolor[rgb]{0.73,0.40,0.53}{##1}}}
\expandafter\def\csname PY@tok@se\endcsname{\let\PY@bf=\textbf\def\PY@tc##1{\textcolor[rgb]{0.73,0.40,0.13}{##1}}}
\expandafter\def\csname PY@tok@sr\endcsname{\def\PY@tc##1{\textcolor[rgb]{0.73,0.40,0.53}{##1}}}
\expandafter\def\csname PY@tok@ss\endcsname{\def\PY@tc##1{\textcolor[rgb]{0.10,0.09,0.49}{##1}}}
\expandafter\def\csname PY@tok@sx\endcsname{\def\PY@tc##1{\textcolor[rgb]{0.00,0.50,0.00}{##1}}}
\expandafter\def\csname PY@tok@m\endcsname{\def\PY@tc##1{\textcolor[rgb]{0.40,0.40,0.40}{##1}}}
\expandafter\def\csname PY@tok@gh\endcsname{\let\PY@bf=\textbf\def\PY@tc##1{\textcolor[rgb]{0.00,0.00,0.50}{##1}}}
\expandafter\def\csname PY@tok@gu\endcsname{\let\PY@bf=\textbf\def\PY@tc##1{\textcolor[rgb]{0.50,0.00,0.50}{##1}}}
\expandafter\def\csname PY@tok@gd\endcsname{\def\PY@tc##1{\textcolor[rgb]{0.63,0.00,0.00}{##1}}}
\expandafter\def\csname PY@tok@gi\endcsname{\def\PY@tc##1{\textcolor[rgb]{0.00,0.63,0.00}{##1}}}
\expandafter\def\csname PY@tok@gr\endcsname{\def\PY@tc##1{\textcolor[rgb]{1.00,0.00,0.00}{##1}}}
\expandafter\def\csname PY@tok@ge\endcsname{\let\PY@it=\textit}
\expandafter\def\csname PY@tok@gs\endcsname{\let\PY@bf=\textbf}
\expandafter\def\csname PY@tok@gp\endcsname{\let\PY@bf=\textbf\def\PY@tc##1{\textcolor[rgb]{0.00,0.00,0.50}{##1}}}
\expandafter\def\csname PY@tok@go\endcsname{\def\PY@tc##1{\textcolor[rgb]{0.53,0.53,0.53}{##1}}}
\expandafter\def\csname PY@tok@gt\endcsname{\def\PY@tc##1{\textcolor[rgb]{0.00,0.27,0.87}{##1}}}
\expandafter\def\csname PY@tok@err\endcsname{\def\PY@bc##1{\setlength{\fboxsep}{0pt}\fcolorbox[rgb]{1.00,0.00,0.00}{1,1,1}{\strut ##1}}}
\expandafter\def\csname PY@tok@kc\endcsname{\let\PY@bf=\textbf\def\PY@tc##1{\textcolor[rgb]{0.00,0.50,0.00}{##1}}}
\expandafter\def\csname PY@tok@kd\endcsname{\let\PY@bf=\textbf\def\PY@tc##1{\textcolor[rgb]{0.00,0.50,0.00}{##1}}}
\expandafter\def\csname PY@tok@kn\endcsname{\let\PY@bf=\textbf\def\PY@tc##1{\textcolor[rgb]{0.00,0.50,0.00}{##1}}}
\expandafter\def\csname PY@tok@kr\endcsname{\let\PY@bf=\textbf\def\PY@tc##1{\textcolor[rgb]{0.00,0.50,0.00}{##1}}}
\expandafter\def\csname PY@tok@bp\endcsname{\def\PY@tc##1{\textcolor[rgb]{0.00,0.50,0.00}{##1}}}
\expandafter\def\csname PY@tok@fm\endcsname{\def\PY@tc##1{\textcolor[rgb]{0.00,0.00,1.00}{##1}}}
\expandafter\def\csname PY@tok@vc\endcsname{\def\PY@tc##1{\textcolor[rgb]{0.10,0.09,0.49}{##1}}}
\expandafter\def\csname PY@tok@vg\endcsname{\def\PY@tc##1{\textcolor[rgb]{0.10,0.09,0.49}{##1}}}
\expandafter\def\csname PY@tok@vi\endcsname{\def\PY@tc##1{\textcolor[rgb]{0.10,0.09,0.49}{##1}}}
\expandafter\def\csname PY@tok@vm\endcsname{\def\PY@tc##1{\textcolor[rgb]{0.10,0.09,0.49}{##1}}}
\expandafter\def\csname PY@tok@sa\endcsname{\def\PY@tc##1{\textcolor[rgb]{0.73,0.13,0.13}{##1}}}
\expandafter\def\csname PY@tok@sb\endcsname{\def\PY@tc##1{\textcolor[rgb]{0.73,0.13,0.13}{##1}}}
\expandafter\def\csname PY@tok@sc\endcsname{\def\PY@tc##1{\textcolor[rgb]{0.73,0.13,0.13}{##1}}}
\expandafter\def\csname PY@tok@dl\endcsname{\def\PY@tc##1{\textcolor[rgb]{0.73,0.13,0.13}{##1}}}
\expandafter\def\csname PY@tok@s2\endcsname{\def\PY@tc##1{\textcolor[rgb]{0.73,0.13,0.13}{##1}}}
\expandafter\def\csname PY@tok@sh\endcsname{\def\PY@tc##1{\textcolor[rgb]{0.73,0.13,0.13}{##1}}}
\expandafter\def\csname PY@tok@s1\endcsname{\def\PY@tc##1{\textcolor[rgb]{0.73,0.13,0.13}{##1}}}
\expandafter\def\csname PY@tok@mb\endcsname{\def\PY@tc##1{\textcolor[rgb]{0.40,0.40,0.40}{##1}}}
\expandafter\def\csname PY@tok@mf\endcsname{\def\PY@tc##1{\textcolor[rgb]{0.40,0.40,0.40}{##1}}}
\expandafter\def\csname PY@tok@mh\endcsname{\def\PY@tc##1{\textcolor[rgb]{0.40,0.40,0.40}{##1}}}
\expandafter\def\csname PY@tok@mi\endcsname{\def\PY@tc##1{\textcolor[rgb]{0.40,0.40,0.40}{##1}}}
\expandafter\def\csname PY@tok@il\endcsname{\def\PY@tc##1{\textcolor[rgb]{0.40,0.40,0.40}{##1}}}
\expandafter\def\csname PY@tok@mo\endcsname{\def\PY@tc##1{\textcolor[rgb]{0.40,0.40,0.40}{##1}}}
\expandafter\def\csname PY@tok@ch\endcsname{\let\PY@it=\textit\def\PY@tc##1{\textcolor[rgb]{0.25,0.50,0.50}{##1}}}
\expandafter\def\csname PY@tok@cm\endcsname{\let\PY@it=\textit\def\PY@tc##1{\textcolor[rgb]{0.25,0.50,0.50}{##1}}}
\expandafter\def\csname PY@tok@cpf\endcsname{\let\PY@it=\textit\def\PY@tc##1{\textcolor[rgb]{0.25,0.50,0.50}{##1}}}
\expandafter\def\csname PY@tok@c1\endcsname{\let\PY@it=\textit\def\PY@tc##1{\textcolor[rgb]{0.25,0.50,0.50}{##1}}}
\expandafter\def\csname PY@tok@cs\endcsname{\let\PY@it=\textit\def\PY@tc##1{\textcolor[rgb]{0.25,0.50,0.50}{##1}}}

\def\PYZbs{\char`\\}
\def\PYZus{\char`\_}
\def\PYZob{\char`\{}
\def\PYZcb{\char`\}}
\def\PYZca{\char`\^}
\def\PYZam{\char`\&}
\def\PYZlt{\char`\<}
\def\PYZgt{\char`\>}
\def\PYZsh{\char`\#}
\def\PYZpc{\char`\%}
\def\PYZdl{\char`\$}
\def\PYZhy{\char`\-}
\def\PYZsq{\char`\'}
\def\PYZdq{\char`\"}
\def\PYZti{\char`\~}
% for compatibility with earlier versions
\def\PYZat{@}
\def\PYZlb{[}
\def\PYZrb{]}
\makeatother


    % Exact colors from NB
    \definecolor{incolor}{rgb}{0.0, 0.0, 0.5}
    \definecolor{outcolor}{rgb}{0.545, 0.0, 0.0}



    
    % Prevent overflowing lines due to hard-to-break entities
    \sloppy 
    % Setup hyperref package
    \hypersetup{
      breaklinks=true,  % so long urls are correctly broken across lines
      colorlinks=true,
      urlcolor=urlcolor,
      linkcolor=linkcolor,
      citecolor=citecolor,
      }
    % Slightly bigger margins than the latex defaults
    
    \geometry{verbose,tmargin=1in,bmargin=1in,lmargin=1in,rmargin=1in}
    
    

    \begin{document}
    
    
    \maketitle
    
    

    
    \hypertarget{assignment-1}{%
\section{Assignment 1}\label{assignment-1}}

\hypertarget{eecs-491}{%
\subsubsection{EECS 491}\label{eecs-491}}

\hypertarget{tristan-maidment---tdm47}{%
\subsubsection{Tristan Maidment -
TDM47}\label{tristan-maidment---tdm47}}

\hypertarget{goal}{%
\paragraph{Goal}\label{goal}}

My goal for this assignment is to use Bayes' Rule to be able to classify
the loss condition for packets retransmitted by the TCP protocol.

Traditionally, TCP has two conditions for re-transmitting packets. TCP
protocol classifies the lack of an acknowledgement after a certain time
as a ``timeout''. Upon receiving three acknowledgements for the same
packet in a row, the packet is ``fast-retransmited'', or instantly sent.
Both of these events are related to a continuous variable, RTT that
defines the time taken for a packet to be sent to a receiver, and an
acknowledgement made. Since both methods require a form of induciton for
packet retransmission, (when it is most probable that a packet should be
re-sent,) they have natural variance. However, I should be able to
predict the retransmission method for a given packet is lost, given RTT
data of timeout and fast-retransmit events.

\hypertarget{method}{%
\paragraph{Method}\label{method}}

To classify re-transmitted packets I will use the following adaptation
of Bayes' Rule:

\texttt{P(timeout\textbar{}RTT)\ =\ P(RTT\textbar{}timeout)P(timeout)}

The fast-retransmit and timeout events are mutually exclusive - only one
event may occur per packet. Due to this, I am able to conclude:

\texttt{P(fast-retransmit\textbar{}RTT)\ =\ 1\ -\ P(timeout\textbar{}RTT)}

Alternatively, I can calculate \texttt{P(fast-retransmit\textbar{}RTT)}
via it's own distribution, and normalize the two.

I will use the RTT data of timeout and fast-retransmit events to create
two likelihood distributions for both variables.

I will be looking to determine the possible ``hidden variable'' linking
the fast-retransmit and timeout events to a packet retransmission event.

    \hypertarget{code}{%
\paragraph{Code}\label{code}}

First, I import the dependencies used for this assignment.

Please note that I have not used Python before, and a fair amount of
this code is modifications of example code.

    \begin{Verbatim}[commandchars=\\\{\}]
{\color{incolor}In [{\color{incolor}2}]:} \PY{k+kn}{import} \PY{n+nn}{csv}
        \PY{k+kn}{import} \PY{n+nn}{matplotlib}\PY{n+nn}{.}\PY{n+nn}{pyplot} \PY{k}{as} \PY{n+nn}{plt}
        \PY{k+kn}{import} \PY{n+nn}{numpy} \PY{k}{as} \PY{n+nn}{np}
        \PY{k+kn}{import} \PY{n+nn}{scipy}\PY{n+nn}{.}\PY{n+nn}{stats} \PY{k}{as} \PY{n+nn}{stats}
        \PY{k+kn}{import} \PY{n+nn}{math}
        \PY{o}{\PYZpc{}}\PY{k}{matplotlib} inline
\end{Verbatim}


    To acquire data for this assignment, I captured the network traffic from
my laptop using the program WireShark.

Due to limitations in my ability to decipher \emph{.pcap} files and find
packet RTT, I used a trick learned in EECS 325 Computer Networks I -
count the number of hops the packet took from source to destination.
Since the source packet defines the TTL, I only observed packets with
TTLs lower than the default value of 64, in order to prevent problems
with packets that have higher TTLs set.

I define three lists to hold the TTLs for each respective event.

    \begin{Verbatim}[commandchars=\\\{\}]
{\color{incolor}In [{\color{incolor}3}]:} \PY{n}{hops} \PY{o}{=} \PY{p}{[}\PY{p}{]}
        \PY{n}{hops\PYZus{}retrans} \PY{o}{=} \PY{p}{[}\PY{p}{]}
        \PY{n}{hops\PYZus{}timeout} \PY{o}{=} \PY{p}{[}\PY{p}{]}
\end{Verbatim}


    To get TTL data, I had to create a custom dataset using the Wireshark
output. It required me to format the TTLs and packet information into a
CSV file, which I parsed for information.

I classified each packet using the packet information that WireShark
provides. I did this by reading through the output to find how the
retransmission events were classified.

\emph{Note that it took me multiple attempts (with 100000+ packets each
time) to acquire data with enough fast-retransmisison events.}

    \begin{Verbatim}[commandchars=\\\{\}]
{\color{incolor}In [{\color{incolor}4}]:} \PY{k}{with} \PY{n+nb}{open}\PY{p}{(}\PY{l+s+s1}{\PYZsq{}}\PY{l+s+s1}{TTL3.csv}\PY{l+s+s1}{\PYZsq{}}\PY{p}{,} \PY{n}{newline}\PY{o}{=}\PY{l+s+s1}{\PYZsq{}}\PY{l+s+s1}{\PYZsq{}}\PY{p}{)} \PY{k}{as} \PY{n}{csvfile}\PY{p}{:}
            \PY{n}{ttls} \PY{o}{=} \PY{n}{csv}\PY{o}{.}\PY{n}{reader}\PY{p}{(}\PY{n}{csvfile}\PY{p}{,} \PY{n}{delimiter}\PY{o}{=}\PY{l+s+s1}{\PYZsq{}}\PY{l+s+s1}{,}\PY{l+s+s1}{\PYZsq{}}\PY{p}{,} \PY{n}{quotechar}\PY{o}{=}\PY{l+s+s1}{\PYZsq{}}\PY{l+s+s1}{\PYZdq{}}\PY{l+s+s1}{\PYZsq{}}\PY{p}{)}
            \PY{k}{for} \PY{n}{row} \PY{o+ow}{in} \PY{n}{ttls}\PY{p}{:}
                \PY{k}{if} \PY{n}{row}\PY{p}{[}\PY{l+m+mi}{1}\PY{p}{]}\PY{p}{:}
                    \PY{n}{ttl} \PY{o}{=} \PY{l+m+mi}{64} \PY{o}{\PYZhy{}} \PY{n+nb}{int}\PY{p}{(}\PY{n}{row}\PY{p}{[}\PY{l+m+mi}{1}\PY{p}{]}\PY{p}{)}
                    \PY{c+c1}{\PYZsh{}I removed all TTL that are below 0.  }
                    \PY{c+c1}{\PYZsh{}These TTL are either set at a value higher than 64 by the sender,}
                    \PY{c+c1}{\PYZsh{}or packets being sent to local devices on my home network. (TTL = 0)}
                    \PY{k}{if} \PY{n}{ttl} \PY{o}{\PYZgt{}} \PY{l+m+mi}{0}\PY{p}{:}
                        \PY{k}{if} \PY{l+s+s2}{\PYZdq{}}\PY{l+s+s2}{[TCP Retransmission]}\PY{l+s+s2}{\PYZdq{}} \PY{o+ow}{in} \PY{n}{row}\PY{p}{[}\PY{l+m+mi}{2}\PY{p}{]}\PY{p}{:}
                            \PY{n}{hops\PYZus{}timeout}\PY{o}{.}\PY{n}{append}\PY{p}{(}\PY{n}{ttl}\PY{p}{)}
                        \PY{k}{elif} \PY{l+s+s2}{\PYZdq{}}\PY{l+s+s2}{[TCP Fast Retransmission]}\PY{l+s+s2}{\PYZdq{}} \PY{o+ow}{in} \PY{n}{row}\PY{p}{[}\PY{l+m+mi}{2}\PY{p}{]}\PY{p}{:}
                            \PY{n}{hops\PYZus{}retrans}\PY{o}{.}\PY{n}{append}\PY{p}{(}\PY{n}{ttl}\PY{p}{)}
                        \PY{k}{else}\PY{p}{:}
                            \PY{n}{hops}\PY{o}{.}\PY{n}{append}\PY{p}{(}\PY{n}{ttl}\PY{p}{)}
\end{Verbatim}


    To see the percentage of packets that were retransmitted, I counted the
number of packets classified for each event, and printed out the
statistics for each event. Unfortunately the amount of retransmitted
packets is very small, but it will have to do.

    \begin{Verbatim}[commandchars=\\\{\}]
{\color{incolor}In [{\color{incolor}5}]:} \PY{n}{pkt\PYZus{}total} \PY{o}{=} \PY{n+nb}{len}\PY{p}{(}\PY{n}{hops}\PY{p}{)} \PY{o}{+} \PY{n+nb}{len}\PY{p}{(}\PY{n}{hops\PYZus{}timeout}\PY{p}{)} \PY{o}{+} \PY{n+nb}{len}\PY{p}{(}\PY{n}{hops\PYZus{}retrans}\PY{p}{)}
        \PY{n}{pkt\PYZus{}succ} \PY{o}{=} \PY{n+nb}{len}\PY{p}{(}\PY{n}{hops}\PY{p}{)}
        \PY{n}{pkt\PYZus{}time} \PY{o}{=} \PY{n+nb}{len}\PY{p}{(}\PY{n}{hops\PYZus{}timeout}\PY{p}{)}
        \PY{n}{pkt\PYZus{}fast} \PY{o}{=} \PY{n+nb}{len}\PY{p}{(}\PY{n}{hops\PYZus{}retrans}\PY{p}{)}
        \PY{n+nb}{print}\PY{p}{(}\PY{l+s+s2}{\PYZdq{}}\PY{l+s+s2}{Packets successful}\PY{l+s+s2}{\PYZdq{}}\PY{p}{,} \PY{n}{pkt\PYZus{}succ}\PY{p}{)}
        \PY{n+nb}{print}\PY{p}{(}\PY{l+s+s2}{\PYZdq{}}\PY{l+s+s2}{Packets timed out}\PY{l+s+s2}{\PYZdq{}}\PY{p}{,} \PY{n}{pkt\PYZus{}time}\PY{p}{)}
        \PY{n+nb}{print}\PY{p}{(}\PY{l+s+s2}{\PYZdq{}}\PY{l+s+s2}{Packets fast\PYZhy{}retransmitted}\PY{l+s+s2}{\PYZdq{}}\PY{p}{,} \PY{n}{pkt\PYZus{}fast}\PY{p}{)}
        \PY{n+nb}{print}\PY{p}{(}\PY{l+s+s2}{\PYZdq{}}\PY{l+s+s2}{Percent timed out}\PY{l+s+s2}{\PYZdq{}}\PY{p}{,} \PY{n}{pkt\PYZus{}time}\PY{o}{/}\PY{n}{pkt\PYZus{}total}\PY{p}{)}
        \PY{n+nb}{print}\PY{p}{(}\PY{l+s+s2}{\PYZdq{}}\PY{l+s+s2}{Percent fast\PYZus{}retransmitted}\PY{l+s+s2}{\PYZdq{}}\PY{p}{,} \PY{n}{pkt\PYZus{}fast}\PY{o}{/}\PY{n}{pkt\PYZus{}total}\PY{p}{)}
        \PY{n+nb}{print}\PY{p}{(}\PY{l+s+s2}{\PYZdq{}}\PY{l+s+s2}{Packets total}\PY{l+s+s2}{\PYZdq{}}\PY{p}{,} \PY{n}{pkt\PYZus{}total}\PY{p}{)}
\end{Verbatim}


    \begin{Verbatim}[commandchars=\\\{\}]
Packets successful 104546
Packets timed out 73
Packets fast-retransmitted 43
Percent timed out 0.0006974833272821081
Percent fast\_retransmitted 0.0004108463434675431
Packets total 104662

    \end{Verbatim}

    I created a plot for the retransmission events and packets that were
properly acknowledged.

In addition, I calculated the mean, standard deviation, and variance of
both events, as I am assuming a normal distribution.

    \begin{Verbatim}[commandchars=\\\{\}]
{\color{incolor}In [{\color{incolor}6}]:} \PY{n}{mu}\PY{p}{,} \PY{n}{std} \PY{o}{=} \PY{n}{stats}\PY{o}{.}\PY{n}{norm}\PY{o}{.}\PY{n}{fit}\PY{p}{(}\PY{n}{hops}\PY{p}{)}
        \PY{n}{plt}\PY{o}{.}\PY{n}{hist}\PY{p}{(}\PY{n}{hops}\PY{p}{,} \PY{n}{normed}\PY{o}{=}\PY{k+kc}{True}\PY{p}{)}
        \PY{n}{xmin}\PY{p}{,} \PY{n}{xmax} \PY{o}{=} \PY{n}{plt}\PY{o}{.}\PY{n}{xlim}\PY{p}{(}\PY{p}{)}
        \PY{n}{x} \PY{o}{=} \PY{n}{np}\PY{o}{.}\PY{n}{linspace}\PY{p}{(}\PY{n}{xmin}\PY{p}{,} \PY{n}{xmax}\PY{p}{,} \PY{l+m+mi}{100}\PY{p}{)}
        \PY{n}{p} \PY{o}{=} \PY{n}{stats}\PY{o}{.}\PY{n}{norm}\PY{o}{.}\PY{n}{pdf}\PY{p}{(}\PY{n}{x}\PY{p}{,} \PY{n}{mu}\PY{p}{,} \PY{n}{std}\PY{p}{)}
        \PY{n}{plt}\PY{o}{.}\PY{n}{plot}\PY{p}{(}\PY{n}{x}\PY{p}{,} \PY{n}{p}\PY{p}{)}
        
        \PY{n+nb}{print}\PY{p}{(}\PY{l+s+s2}{\PYZdq{}}\PY{l+s+s2}{Successfully Transmitted}\PY{l+s+s2}{\PYZdq{}}\PY{p}{)}
        \PY{n+nb}{print}\PY{p}{(}\PY{l+s+s2}{\PYZdq{}}\PY{l+s+s2}{Mean}\PY{l+s+s2}{\PYZdq{}}\PY{p}{,} \PY{n}{np}\PY{o}{.}\PY{n}{mean}\PY{p}{(}\PY{n}{hops}\PY{p}{)}\PY{p}{)}
        \PY{n+nb}{print}\PY{p}{(}\PY{l+s+s2}{\PYZdq{}}\PY{l+s+s2}{STD}\PY{l+s+s2}{\PYZdq{}}\PY{p}{,} \PY{n}{np}\PY{o}{.}\PY{n}{std}\PY{p}{(}\PY{n}{hops}\PY{p}{)}\PY{p}{)}
        \PY{n+nb}{print}\PY{p}{(}\PY{l+s+s2}{\PYZdq{}}\PY{l+s+s2}{VAR}\PY{l+s+s2}{\PYZdq{}}\PY{p}{,} \PY{n}{np}\PY{o}{.}\PY{n}{std}\PY{p}{(}\PY{n}{hops}\PY{p}{)} \PY{o}{*} \PY{n}{np}\PY{o}{.}\PY{n}{std}\PY{p}{(}\PY{n}{hops}\PY{p}{)}\PY{p}{)}
\end{Verbatim}


    \begin{Verbatim}[commandchars=\\\{\}]
Successfully Transmitted
Mean 10.172871271975973
STD 3.31149695250862
VAR 10.966012066473878

    \end{Verbatim}

    \begin{center}
    \adjustimage{max size={0.9\linewidth}{0.9\paperheight}}{output_10_1.png}
    \end{center}
    { \hspace*{\fill} \\}
    
    \begin{Verbatim}[commandchars=\\\{\}]
{\color{incolor}In [{\color{incolor}7}]:} \PY{n}{mu}\PY{p}{,} \PY{n}{std} \PY{o}{=} \PY{n}{stats}\PY{o}{.}\PY{n}{norm}\PY{o}{.}\PY{n}{fit}\PY{p}{(}\PY{n}{hops\PYZus{}timeout}\PY{p}{)}
        \PY{n}{plt}\PY{o}{.}\PY{n}{hist}\PY{p}{(}\PY{n}{hops\PYZus{}timeout}\PY{p}{,} \PY{n}{normed}\PY{o}{=}\PY{k+kc}{True}\PY{p}{)}
        \PY{n}{p} \PY{o}{=} \PY{n}{stats}\PY{o}{.}\PY{n}{norm}\PY{o}{.}\PY{n}{pdf}\PY{p}{(}\PY{n}{x}\PY{p}{,} \PY{n}{mu}\PY{p}{,} \PY{n}{std}\PY{p}{)}
        \PY{n}{plt}\PY{o}{.}\PY{n}{plot}\PY{p}{(}\PY{n}{x}\PY{p}{,} \PY{n}{p}\PY{p}{)}
        
        \PY{n+nb}{print}\PY{p}{(}\PY{l+s+s2}{\PYZdq{}}\PY{l+s+s2}{Timeout}\PY{l+s+s2}{\PYZdq{}}\PY{p}{)}
        \PY{n+nb}{print}\PY{p}{(}\PY{l+s+s2}{\PYZdq{}}\PY{l+s+s2}{Mean}\PY{l+s+s2}{\PYZdq{}}\PY{p}{,} \PY{n}{np}\PY{o}{.}\PY{n}{mean}\PY{p}{(}\PY{n}{hops\PYZus{}timeout}\PY{p}{)}\PY{p}{)}
        \PY{n+nb}{print}\PY{p}{(}\PY{l+s+s2}{\PYZdq{}}\PY{l+s+s2}{STD}\PY{l+s+s2}{\PYZdq{}}\PY{p}{,} \PY{n}{np}\PY{o}{.}\PY{n}{std}\PY{p}{(}\PY{n}{hops\PYZus{}timeout}\PY{p}{)}\PY{p}{)}
        \PY{n+nb}{print}\PY{p}{(}\PY{l+s+s2}{\PYZdq{}}\PY{l+s+s2}{VAR}\PY{l+s+s2}{\PYZdq{}}\PY{p}{,} \PY{n}{np}\PY{o}{.}\PY{n}{std}\PY{p}{(}\PY{n}{hops\PYZus{}timeout}\PY{p}{)} \PY{o}{*} \PY{n}{np}\PY{o}{.}\PY{n}{std}\PY{p}{(}\PY{n}{hops\PYZus{}timeout}\PY{p}{)}\PY{p}{)}
\end{Verbatim}


    \begin{Verbatim}[commandchars=\\\{\}]
Timeout
Mean 11.232876712328768
STD 4.5616644040837855
VAR 20.808782135485078

    \end{Verbatim}

    \begin{center}
    \adjustimage{max size={0.9\linewidth}{0.9\paperheight}}{output_11_1.png}
    \end{center}
    { \hspace*{\fill} \\}
    
    \begin{Verbatim}[commandchars=\\\{\}]
{\color{incolor}In [{\color{incolor}8}]:} \PY{n}{mu}\PY{p}{,} \PY{n}{std} \PY{o}{=} \PY{n}{stats}\PY{o}{.}\PY{n}{norm}\PY{o}{.}\PY{n}{fit}\PY{p}{(}\PY{n}{hops\PYZus{}retrans}\PY{p}{)}
        \PY{n}{plt}\PY{o}{.}\PY{n}{hist}\PY{p}{(}\PY{n}{hops\PYZus{}retrans}\PY{p}{,} \PY{n}{normed}\PY{o}{=}\PY{k+kc}{True}\PY{p}{)}
        \PY{n}{p} \PY{o}{=} \PY{n}{stats}\PY{o}{.}\PY{n}{norm}\PY{o}{.}\PY{n}{pdf}\PY{p}{(}\PY{n}{x}\PY{p}{,} \PY{n}{mu}\PY{p}{,} \PY{n}{std}\PY{p}{)}
        \PY{n}{plt}\PY{o}{.}\PY{n}{plot}\PY{p}{(}\PY{n}{x}\PY{p}{,} \PY{n}{p}\PY{p}{)}
        
        \PY{n+nb}{print}\PY{p}{(}\PY{l+s+s2}{\PYZdq{}}\PY{l+s+s2}{Fast\PYZhy{}Retransmit}\PY{l+s+s2}{\PYZdq{}}\PY{p}{)}
        \PY{n+nb}{print}\PY{p}{(}\PY{l+s+s2}{\PYZdq{}}\PY{l+s+s2}{Mean}\PY{l+s+s2}{\PYZdq{}}\PY{p}{,} \PY{n}{np}\PY{o}{.}\PY{n}{mean}\PY{p}{(}\PY{n}{hops\PYZus{}retrans}\PY{p}{)}\PY{p}{)}
        \PY{n+nb}{print}\PY{p}{(}\PY{l+s+s2}{\PYZdq{}}\PY{l+s+s2}{STD}\PY{l+s+s2}{\PYZdq{}}\PY{p}{,} \PY{n}{np}\PY{o}{.}\PY{n}{std}\PY{p}{(}\PY{n}{hops\PYZus{}retrans}\PY{p}{)}\PY{p}{)}
        \PY{n+nb}{print}\PY{p}{(}\PY{l+s+s2}{\PYZdq{}}\PY{l+s+s2}{VAR}\PY{l+s+s2}{\PYZdq{}}\PY{p}{,} \PY{n}{np}\PY{o}{.}\PY{n}{std}\PY{p}{(}\PY{n}{hops\PYZus{}retrans}\PY{p}{)} \PY{o}{*} \PY{n}{np}\PY{o}{.}\PY{n}{std}\PY{p}{(}\PY{n}{hops\PYZus{}retrans}\PY{p}{)}\PY{p}{)}
\end{Verbatim}


    \begin{Verbatim}[commandchars=\\\{\}]
Fast-Retransmit
Mean 14.511627906976743
STD 1.70324031847447
VAR 2.9010275824770138

    \end{Verbatim}

    \begin{center}
    \adjustimage{max size={0.9\linewidth}{0.9\paperheight}}{output_12_1.png}
    \end{center}
    { \hspace*{\fill} \\}
    
    It's very interesting to see the concentration of fast retransmits at 15
hops. Unfortunately the timeout event distribution is similar to that of
the successful packets.

I then define the prior, P(loss). This is a constant as it is the
overall packet loss of the model.

    \begin{Verbatim}[commandchars=\\\{\}]
{\color{incolor}In [{\color{incolor}9}]:} \PY{n}{packet\PYZus{}loss} \PY{o}{=} \PY{p}{(}\PY{n}{pkt\PYZus{}time}\PY{o}{+}\PY{n}{pkt\PYZus{}fast}\PY{p}{)}\PY{o}{/}\PY{n}{pkt\PYZus{}total}
\end{Verbatim}


    To determine the probability of a specific TTL given a certain
retransmission event, I use the PDF from the likelihood distributions.

    \begin{Verbatim}[commandchars=\\\{\}]
{\color{incolor}In [{\color{incolor}10}]:} \PY{k}{def} \PY{n+nf}{likelihood\PYZus{}timeout}\PY{p}{(}\PY{n}{TTL}\PY{p}{)}\PY{p}{:}
             \PY{n}{mu}\PY{p}{,} \PY{n}{std} \PY{o}{=} \PY{n}{stats}\PY{o}{.}\PY{n}{norm}\PY{o}{.}\PY{n}{fit}\PY{p}{(}\PY{n}{hops\PYZus{}timeout}\PY{p}{)}
             \PY{k}{return} \PY{n}{stats}\PY{o}{.}\PY{n}{norm}\PY{p}{(}\PY{n}{mu}\PY{p}{,} \PY{n}{std}\PY{p}{)}\PY{o}{.}\PY{n}{pdf}\PY{p}{(}\PY{n}{TTL}\PY{p}{)}
         
         \PY{k}{def} \PY{n+nf}{likelihood\PYZus{}retransmit}\PY{p}{(}\PY{n}{TTL}\PY{p}{)}\PY{p}{:}
             \PY{n}{mu}\PY{p}{,} \PY{n}{std} \PY{o}{=} \PY{n}{stats}\PY{o}{.}\PY{n}{norm}\PY{o}{.}\PY{n}{fit}\PY{p}{(}\PY{n}{hops\PYZus{}retrans}\PY{p}{)}
             \PY{k}{return} \PY{n}{stats}\PY{o}{.}\PY{n}{norm}\PY{p}{(}\PY{n}{mu}\PY{p}{,} \PY{n}{std}\PY{p}{)}\PY{o}{.}\PY{n}{pdf}\PY{p}{(}\PY{n}{TTL}\PY{p}{)}
\end{Verbatim}


    To define the posterior, we take the likelihood and multiply it by the
prior.

    \begin{Verbatim}[commandchars=\\\{\}]
{\color{incolor}In [{\color{incolor}11}]:} \PY{k}{def} \PY{n+nf}{posterior\PYZus{}timeout}\PY{p}{(}\PY{n}{TTL}\PY{p}{)}\PY{p}{:}
             \PY{k}{return} \PY{n}{likelihood\PYZus{}timeout}\PY{p}{(}\PY{n}{TTL}\PY{p}{)} \PY{o}{*} \PY{n}{packet\PYZus{}loss}
         
         \PY{k}{def} \PY{n+nf}{posterior\PYZus{}retransmit}\PY{p}{(}\PY{n}{TTL}\PY{p}{)}\PY{p}{:}
             \PY{k}{return} \PY{n}{likelihood\PYZus{}retransmit}\PY{p}{(}\PY{n}{TTL}\PY{p}{)} \PY{o}{*} \PY{n}{packet\PYZus{}loss}
\end{Verbatim}


    To determine the classification boundaries between fast-retransmit and
timeout, we must normalize the two posterior distributions.

    \begin{Verbatim}[commandchars=\\\{\}]
{\color{incolor}In [{\color{incolor}12}]:} \PY{k}{def} \PY{n+nf}{normalize}\PY{p}{(}\PY{n}{val}\PY{p}{)}\PY{p}{:}
             \PY{n}{norm\PYZus{}const} \PY{o}{=} \PY{n}{posterior\PYZus{}timeout}\PY{p}{(}\PY{n}{val}\PY{p}{)} \PY{o}{+} \PY{n}{posterior\PYZus{}retransmit}\PY{p}{(}\PY{n}{val}\PY{p}{)}
             \PY{k}{return} \PY{p}{(}\PY{n}{posterior\PYZus{}timeout}\PY{p}{(}\PY{n}{val}\PY{p}{)} \PY{o}{/} \PY{n}{norm\PYZus{}const}\PY{p}{)}\PY{p}{,} \PY{p}{(}\PY{n}{posterior\PYZus{}retransmit}\PY{p}{(}\PY{n}{val}\PY{p}{)} \PY{o}{/} \PY{n}{norm\PYZus{}const}\PY{p}{)}
\end{Verbatim}


    I then graphed the decision boundary for both events, superimposed on
each other. We can see that around 15 TTLs, Bayes' Inference picks the
fast-retransmission event over the timeout event. The likelihood
distribution for the fast-retransmit event supports this.

    \begin{Verbatim}[commandchars=\\\{\}]
{\color{incolor}In [{\color{incolor}13}]:} \PY{n}{xaxis} \PY{o}{=} \PY{p}{[}\PY{p}{]}
         \PY{n}{timeout\PYZus{}prob} \PY{o}{=} \PY{p}{[}\PY{p}{]}
         \PY{n}{retransmit\PYZus{}prob} \PY{o}{=} \PY{p}{[}\PY{p}{]}
         
         \PY{k}{for} \PY{n}{z} \PY{o+ow}{in} \PY{n+nb}{range}\PY{p}{(}\PY{l+m+mi}{0}\PY{p}{,} \PY{n}{math}\PY{o}{.}\PY{n}{ceil}\PY{p}{(}\PY{n}{xmax}\PY{p}{)}\PY{p}{)}\PY{p}{:}
             \PY{n}{xaxis}\PY{o}{.}\PY{n}{append}\PY{p}{(}\PY{n}{z}\PY{p}{)}
             \PY{n}{timeout\PYZus{}prob}\PY{o}{.}\PY{n}{append}\PY{p}{(}\PY{n}{normalize}\PY{p}{(}\PY{n}{z}\PY{p}{)}\PY{p}{[}\PY{l+m+mi}{0}\PY{p}{]}\PY{p}{)}
             \PY{n}{retransmit\PYZus{}prob}\PY{o}{.}\PY{n}{append}\PY{p}{(}\PY{n}{normalize}\PY{p}{(}\PY{n}{z}\PY{p}{)}\PY{p}{[}\PY{l+m+mi}{1}\PY{p}{]}\PY{p}{)}
         \PY{n}{plt}\PY{o}{.}\PY{n}{plot}\PY{p}{(}\PY{n}{xaxis}\PY{p}{,} \PY{n}{timeout\PYZus{}prob}\PY{p}{)}
         \PY{n}{plt}\PY{o}{.}\PY{n}{plot}\PY{p}{(}\PY{n}{xaxis}\PY{p}{,} \PY{n}{retransmit\PYZus{}prob}\PY{p}{)}
         
         \PY{n+nb}{print}\PY{p}{(}\PY{l+s+s2}{\PYZdq{}}\PY{l+s+s2}{Bayesian Inference Decision w/ Respect to TTL}\PY{l+s+s2}{\PYZdq{}}\PY{p}{)}
\end{Verbatim}


    \begin{Verbatim}[commandchars=\\\{\}]
Bayesian Inference Decision w/ Respect to TTL

    \end{Verbatim}

    \begin{center}
    \adjustimage{max size={0.9\linewidth}{0.9\paperheight}}{output_22_1.png}
    \end{center}
    { \hspace*{\fill} \\}
    
    \hypertarget{conclusion}{%
\subsubsection{Conclusion}\label{conclusion}}

Overall, I believe that I reached my goal in terms of being able to
classify what sort of retransmission mechanism a lost packet will take.
Due to limitations in the amount of data I was able to require, I
believe that this representation is not accurate of actual TCP
throughput, but I believe that it accurately classify's the model I
acquired.


    % Add a bibliography block to the postdoc
    
    
    
    \end{document}
